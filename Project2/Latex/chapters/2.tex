\section{Adopted solutions}

\subsection*{Part A: Clustering}
This part is conducted in an unsupervised approach, with the idea of grouping
sentences with clustering algorithms. In particular, we used the
\textbf{sklearn} library, with the AgglomerativeClustering class as
suggested.\\ The main paths to explore in this part are:
\begin{itemize}
    \item \textbf{Number of clusters and used metrics: }  the main challenge here was the correct choice of the \textbf{number of clusters}, because it's a parameter than could have a big impact on the results. Using \textbf{silhouette score } we have solved this problem in a iterative way.
    \item \textbf{Sentences selection: } the second challenge was to select the sentences that would be used to build the summary. Using \textbf{centroids} of each clusters, we were able to build summaries that had more topics and were more representative of the original text.
\end{itemize}

\subsubsection*{Number of clusters and used metrics} 
A problem related to part A is to define a good number of clusters to represent the feature space of the document.\\
In this part, we tried to construct a custom metric taking into account Silhouette score, Calinski-Harabasz score and a function of the number of clusters. The idea was to consider the number of clusters, that can vary in the range $[2,numberOfSentences]$, to avoid high sparse clusters representation with single-sentence clusters, but in the end we noticed that the silhouette score by itself was performing overall better.\\We are leaving the code for the custom metric in the notebook, but it is commented since it was not used in the final version of the code.\\

\subsubsection*{Sentences selection}
Another relevant problem, once completed the clustering part of the project, was to decide the criteria to pick the sentences to use in the summarization. We decided to use the \textbf{centroid} of each cluster and, based on the distance from it, we pick the required number of sentences. We have done some tests on different criteria, but others ideas that we had were not really convincing on summaries, so we kept the centroid distance as metric. 
\subsection*{Part B: Classification}