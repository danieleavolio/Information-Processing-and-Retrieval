\section{Adopted solutions}
Each task of the project presented a set of choices or challenges that we had to overcome. This section of the report summarizes briefly those aspects, describing our solution in a non-exhaustive way. To better understand our proposed system, please refer to the code in the ipynb file.

\subsection{Indexing}
The indexing phase sticks to the Whoosh library implementation, because it was fast and straightforward. The only notable operation is the title removal in documents.\\
We did not implement any scorer function by scratch, since the one already present in the library are performing working in our use case.  
\subsection{Text summarization}
The general idea was to produce a summarization system based on the \emph{extractive approach}. In particular,the task was fulfilled with 3 different IR systems: BM25 scorer, TF-IDF scorer and a BERT-based system, with further tests on Reciprocal Rank Fusion and Maximal Marginal Relevance. \\
\subsubsection*{MB25 and TF-IDF}
BM25 and TF-IDF are two models based on the respective metrics that are querying on the inverted index created in the first step. The score of each sentence is related to the length of it and we prefer the top $k$ scoring sentences. We also have the possibility to get the extracted sentences in the exact order of appearance in the initial document.
\subsubsection*{BERT}
Our BERT model is based on a sentence selective approach in which we prefer the sentences that are closer to the document in the latent space representation produced by BERT. This might represent a problem in contexts with long documents, since the BERT token windows is limited to 512. However, we accept that possible information loss for evaluation purposes.
\subsubsection*{Reciprocal Rank Fusion and Maximal Marginal Relevance}
We also explored the RRF and MMR techniques to improve the quality of the summaries. The first one is a way to combine the results of different systems, in our case BM25 and BERT, while the second one is a way to avoid redundancy in the output summaries.\\ Both of them are limited by important computational times, since the first requires the usage of BERT and the second is an iterative process.

\subsection{Keyword extraction}
We used the \emph{nltk} library to tag words based on the context that they provide in the sentence. We then selected a few relevant tags, like nouns and pronouns. Using TF-IDF, we scored each of the selected word to get the $k$ most relevant ones. 

\subsection{Evaluation}
The used metrics in this phase are precision, recall, F1 score and Mean Average Precision (MAP).
As required in the project description, we conduced a set of evaluation tests to assert the quality of the system, but the complete collection could not be tested for some specific evaluation tasks due to lack of computational power. More on this can be found in the upcoming section, as well as in the proposed notebook.