\documentclass{article}

% Packages for code, figures, and automata
\usepackage{listings} % For code listings
\usepackage{graphicx} % For including figures
\usepackage{tikz}     % For drawing automata
\usepackage{multirow}
\usepackage{array}
\usepackage{amssymb} % Pacchetto per simboli matematici
\usepackage{float}

%colorize lstlisting with language
\usepackage{xcolor}

\usepackage{titlesec}

%import all important packages 

% Configurazione degli stili per tutti i linguaggi
\lstset{
  basicstyle=\ttfamily,
  keywordstyle=\color{blue},
  commentstyle=\color{green},
  stringstyle=\color{red},
  % Altre opzioni
  breaklines=true, showstringspaces=false, emph={label},
  emphstyle={\color{custompurple}}, escapeinside={(*}{*)} }

% Stile globale per tutti i linguaggi
\lstdefinestyle{mystyle}{
  backgroundcolor=\color{white},
  commentstyle=\color{green},
  keywordstyle=\color{blue},
  numberstyle=\tiny\color{gray},
  stringstyle=\color{red},
  basicstyle=\ttfamily\footnotesize,
  breakatwhitespace=false,
  breaklines=true,
  captionpos=b,
  keepspaces=true,
  numbers=left,
  numbersep=5pt,
  showspaces=false,
  showstringspaces=false,
  showtabs=false,
  tabsize=2
}

% Impostazioni per tutti i linguaggi
\lstset{style=mystyle}
% Definizione di simboli per subsection e subsubsection
\newcommand{\subsecsymbol}{\textcolor{custompurple}{\rule[0pt]{10pt}{10pt}\hspace{10pt}}}
\newcommand{\subsubsecsymbol}{\textcolor{custompurple}{\textbf{$\blacklozenge$}\hspace{4pt}}}

\titleformat{\section}[block]
{\Huge\bfseries}
{\llap{\textcolor{black}{\rule[-4pt]{10pt}{18pt}\hspace{10pt}}\thesection\hskip 12pt}}
{0pt}
{}
% Definizione di uno stile per \subsection
\titleformat{\subsection}[block]
{\Large\bfseries\color{black}}
{\llap{\subsecsymbol}\thesubsection\hskip 12pt}
{0pt}
{}

% Definizione di uno stile per \subsubsection
\titleformat{\subsubsection}[block]
{\large\bfseries\color{black}}
{\llap{\subsubsecsymbol}\thesubsubsection\hskip 12pt}
{0pt}
{}

%make link clickable
\usepackage{hyperref}
\usepackage{pgfplots}

%use asmath
\usepackage{amsmath}

\usepackage{fancyhdr}
\pagestyle{fancy}
\fancyhf{}
\fancyhead[R]{\nouppercase{\rightmark}}
\fancyfoot[C]{\thepage}

\usepackage{l   istings}
\usepackage{tabularx}

%color link orange
\hypersetup{
  colorlinks=true,
  linkcolor=black,
  filecolor=magenta,
  urlcolor=cyan,
}

\definecolor{custompurple}{HTML}{8b3fff}

% Titolo e autore del documento
\title{Information Processing and Retrieval \\
  \large Part 1 Project Report}
% big test Group08


\author{
  \textbf{Group 08:}\\
  Daniele Avolio,\ ist1111559 \\
  Michele Vitale,\ ist1111558 \\
  Luís Dias,\ ist198557 \\
}

\date{}

\begin{document}

\maketitle

\newpage

% Include your chapters or sections here
\section{Problem Statement}
In this project we handled the task of \textbf{summarizing} and \textbf{extracing keywords} from a set of documents. In particular, 
we were handling documents regarding news from the BBC.
Our dataset is composed both from the plain text of the news and the corresponding summarization, retrieved using state of the art 
techniques. Note that in this part of the project we didn't use the summarization
of the news for guiding our system, but we only used 
it for evaluating the performance of it.

The tasks that were conducted can be explored one by one. Let's start by 
listing them:

\begin{itemize}
    \item \textbf{Indexing}: The creation of a structure 
    that allows to quickly retrieve the documents.
    \item \textbf{Text Summarization}: Given a document, the task is to 
    compute the best set of sentences that are more relevant to the document itself.
    \item \textbf{Keyword Extraction}: Given a document, retrieve the
    most important words that are present in the document.
    \item \textbf{Evaluation}: Given a set of produced summaries $S_p$ and a set of real summaries $S_r$,
    compute the metrics that evaluate the performance of the system.
\end{itemize}

Moreover, for the task of \textbf{Text Summarization} we explored 
some techniques to improve the performance of the system. Namely, 
we tried to use \textbf{Reciprocal Rank Fusion (RRF)}, a technique that
allows to combine the results of different systems in order to
improve the performance of the system itself, and \textbf{Maximal Marginal Relevance (MMR)}, a techniques that
theoretically reduces the redundancy of the produced summaries. \\
This report can not contain all the data and graphs that we produced, so for more complete informations it is strongly suggested to check the comments on the provided notebook.\\
Furthermore,it is important to note that some tasks were computationally too expensive, so they have been conducted on a subset of the dataset. This is the case of the evaluation phase, that was conducted on a subset of 20 documents, and the RRF and MMR tests, that were conducted on a subset of the provided library.


\end{document}
