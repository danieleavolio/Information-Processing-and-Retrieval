\section{Problem Statement}
In this project we handled the task of \textbf{summarizing} and \textbf{extracing keywords} from a set of documents. In particular, 
we were handling documents regarding news from the BBC.
Our dataset is composed both from the plain text of the news and the corresponding summarization, retrieved using state of the art 
techniques. Note that in this part of the project we didn't use the summarization
of the news for guiding our system, but we only used 
it for evaluating the performance of it.

The tasks that were conducted can be explored one by one. Let's start by 
listing them:

\begin{itemize}
    \item \textbf{Indexing}: The creation of a structure 
    that allows to quickly retrieve the documents.
    \item \textbf{Text Summarization}: Given a document, the task is to 
    compute the best set of sentences that are more relevant to the document itself.
    \item \textbf{Keyword Extraction}: Given a document, retrieve the
    most important words that are present in the document.
    \item \textbf{Evaluation}: Given a set of produced summaries $S_p$ and a set of real summaries $S_r$,
    compute the metrics that evaluate the performance of the system.
\end{itemize}

Moreover, for the task of \textbf{Text Summarization} we explored 
some techniques to improve the performance of the system. Namely, 
we tried to use \textbf{Reciprocal Rank Fusion (RRF)}, a technique that
allows to combine the results of different systems in order to
improve the performance of the system itself, and \textbf{Maximal Marginal Relevance (MMR)}, a techniques that
theoretically reduces the redundancy of the produced summaries.

Cose da aggiungere:
\begin{itemize}
    \item Specificare che se mancano alcuni grafici e dati, sono sul notebook
    \item Specificareche il numero di documenti sui quali sono stati tirati fuori i grafici. In particolare, RRF fatto su tot documenti, MMR su tot documenti, comparison con BERT su tot documenti.  Questo appunto per dire che non è sull'intera collection.
\end{itemize}